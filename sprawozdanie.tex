\documentclass[a4paper]{article}
\usepackage[polish]{babel}
\usepackage[OT4]{fontenc}
\usepackage{amsfonts}
\usepackage{amsmath}
\usepackage[utf8]{inputenc}
\usepackage[margin=0.5in]{geometry}
\title{Sprawozdanie sprawdzian PWI}
\date{\today}
\author{Mateusz Marszałek}
\begin{document}
 \maketitle
 $\rho \frac{D\mathbf{u}}{Dt}=\rho (\frac{\partial \mathbf{u}}{\partial u}+\mathbf{u} \cdot \mathbf{\nabla u})=-\nabla \bar{p}+\nabla \{\mu (\nabla \mathbf{u}+(\bigtriangledown \mathbf{u})^T-\frac{2}{3}(\nabla \mathbf{u})\mathbf{I})\}+\rho \mathbf{g}$\\
 \vspace{1 cm}
 
 $\tilde{f}(\xi)=\int_{-\infty}^{\infty}f(x)e^{-2\pi ix\xi}dx$\\ 
 \vspace{1 cm}
 
 $\mathbb{P}(\hat{X_n}-z_{1-{\frac{\alpha}{2}}}\frac{\sigma}{\sqrt{n}}\leq \mathbb{E}X\leq \hat{X_n}+z_{1-\frac{\alpha}{2}}\frac{\sigma}{\sqrt{n}})\approx 1-\alpha$\\
 \vspace{1 cm}
 
 $
  \begin{bmatrix}
   1 & 2 \\
   3 & 4
  \end{bmatrix}
  \oplus
  \begin{bmatrix}
  0 & 5 \\
  6 & 7
  \end{bmatrix}
  =
  \begin{bmatrix}
   1\begin{bmatrix}
   0 & 5\\
   6 & 7
   \end{bmatrix}
   & 2\begin{bmatrix}
   0 & 5\\
   6 & 7
   \end{bmatrix}
   \\
   3\begin{bmatrix}
   0 & 5\\
   6 & 7
   \end{bmatrix}
   & 4\begin{bmatrix}
   0 & 5\\
   6 & 7
   \end{bmatrix}
  \end{bmatrix}
  =
  \begin{bmatrix}
   0 & 5 & 0 & 10 \\
   6 & 7 & 12 & 14 \\
   0 & 15 & 0 & 20 \\
   18 & 21 & 24 & 28
  \end{bmatrix}
$

\section{}
\subsection{}
Dwa klucze generujemy poleceniem ssh-keygen.
\subsection{}
Do przeniesienia klucza używamy 'ssh-copy-id -i' ścieżka do klucza i jego nazwa, a następnie podajemy login i server z którym się łączymy.
\subsection{}
Klucz możemy dodać bezpośrednio do repozytorium lub do całego konta w ustawieniach.
\subsection{}
w pliku config w folderze .ssh trzeba dodać\\ 
\\
Host pwi-sprawdzian\\
	Hostname pwi.ii.uni.wroc.pl\\
	user m323941\\
	ForwardAgent yes\\
	\\
Dzięki temu możemy logować się na server poleceniem
ssh pwi-sprawdzian
\subsection{}
Dodana wyżej linikja 'ForwardAgent yes' sprawia, że możemy używać kluczy z głównego urządzenia na serwerze.
\newpage
\section{}
\subsection{}
Pliki kopiujemy na serwer poleceniem scp ścieżka i nazwa do pliku ora adres serwera zakończony dwukropkiem. Musimy także po scp dodać -r jeśli chcemy wysłać folder.\\
Wygenerowanie klucza na serwerze jest 'brzydkie' gdyż te klucze które tam stworzymy zostaną na serwerze, a nie na naszym komputerze, przez co, ktoś może wziąć te klucze i użyć ich by otrzymać dostęp do naszych usług, kont itp.
\end{document}
